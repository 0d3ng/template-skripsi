%-------------------------------------------------------------------------------
%                            	BAB V
%               		KESIMPULAN DAN SARAN
%-------------------------------------------------------------------------------

\chapter{KESIMPULAN DAN SARAN}

\section{Kesimpulan}
	\begin{enumerate}
		\item Diperlukan pengendalian yang dapat mengendalikan atau mengakuisisi data dari WSN secara jarak jauh.
		\item Solusi pengendalian WSN jarak jauh adalah interoperabilitas dengan protokol Internet.
		\item Namun, Internet gateway untuk WSN biasanya hanya diperuntukkan untuk vendor WSN tertentu.
		\item Dengan AP yang rendah biaya, kita dapat membangun aplikasi sebagai gateway WSN bermacam vendor ke Internet.
		\item Penelitian ini menunjukkan bahwa dengan TP-LINK MR3020 dengan sistem operasi OpenWRT yang terinstal aplikasi-aplikasi pendukung, seperti Python, PySerial, PHP, uHTTPd, dan at, Gateway WSN \emph{multiple vendor} untuk interoperabilitas dengan protokol Internet dapat terwujud.
	\end{enumerate}


\section{Saran}
	\begin{enumerate}
		\item Penelitian selanjutnya dapat menggunakan AP selain TP-LINK MR3020 sebagai gateway dan membandingkannya dengan performa TP-LINK MR3020 yang memiliki \emph{bug} pada extroot.
		\item Penelitian ini menggunakan Python untuk berkomunikasi dengan kanal serial, penelitian selanjutnya dapat menggunakan pendekatan yang lain, seperti membangun aplikasi dengan bahasa C.
		\item Penelitian selanjutnya dapat menggunakan jenis-jenis WSN yang lebih bervariatif.
	\end{enumerate}

	