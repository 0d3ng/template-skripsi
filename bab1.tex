%-------------------------------------------------------------------------------
% 								BAB I
% 							LATAR BELAKANG
%-------------------------------------------------------------------------------

\chapter{LATAR BELAKANG}

\section{Latar Belakang Masalah}
Jaringan sensor nirkabel, atau yang juga dikenal dengan istilah \emph{Wireless Sensor Network} (WSN), adalah jaringan simpul sensor otonom terdistribusi yang dapat berkomunikasi satu sama lain secara nirkabel \cite{wibowo2013wireless,Raluca2008}. WSN tidak hanya digunakan dalam lingkup penelitian tapi juga sudah digunakan serara luas dalam implementasinya di pengamatan lingkungan, fasilitas rumah cerdas, pelacakan alam dan ekologis, pengendalian kualitas dalam teknik industri dan pembangunan, prediksi tanah longsor, dan lain-lain.

Penggunaan WSN untuk sebuah gedung dan perumahan kian populer \cite{Matsuura2007} karena dapat dimanfaatkan untuk berbagai kepentingan. Contoh penerapan WSN dalam rumah yang sangat populer adalah \emph{home automation} yaitu proses automatisasi segala rutinitas yang ada di rumah. Sebagai contoh, sang pemilik rumah harus menyalakan lampu di kala waktu sudah senja dan atau menyalakan pendingin ruangan saat pemilik baru saja pulang dari bekerja. Segala sesuatu yang mungkin untuk diautomatisasi, dapat terealisasi dengan bantuan WSN. Contoh lain penerapan WSN dalam rumah adalah \emph{home surveillance} yaitu pemanfaatan WSN untuk mengawasi tiap sudut rumah secara \emph{realtime}. Dengan ini, sang pemilik rumah tidak perlu lagi khawatir jika rumahnya kurang pengawasan karena mengawasi rumah menjadi semakin mudah dengan bantuan teknologi WSN ini.

Pada umumnya, WSN dikendalikan oleh sebuah \emph{sink node} yang berada dekat pada wilayah jaringan sensornya \cite{Lin2011}. Padahal, pengendalian WSN sebagai fasilitas rumah cerdas acap kali memerlukan pendendalian secara jarak jauh karena sang pemilik rumah tidak selalu berada di dalam rumah.

Selain itu, sebuah \emph{sink node}-pun juga memiliki keterbatasan, terlebih dalam hal menangani beberapa WSN yang berasal dari berbagai \emph{vendor}. Sebagai contoh, sebuah \emph{sink node} keluaran IQRF tidak dapat menangani WSN lain selain dari IQRF. Begitu pula \emph{sink node} produk XBee, misalnya, tidak bisa menangani sama sekali sensor-sensor selain produk XBee.

Lebih jauh lagi, protokol yang digunakan dalam WSN adalah protokol yang sifatnya \emph{proprietary} yang sifatnya tertutup dan tidak mendukung interoperabilitas. Hal ini menyebabkan tiap WSN dari \emph{vendor} yang berbeda tidak dapat saling berkomunikasi.

Protokol Internet menawarkan fleksibilitas dan interoperabilitas WSN dengan piranti lain \cite{Hwang2003}, bahkan dengan piranti selain WSN seperti komputer dan telepon seluler. Namun demikian, protokol Internet boros dalam pengkonsumsian daya listrik. Ada beberapa cara untuk mengintegrasikan IP dengan WSN \cite{Rodrigues2010}, salah satu caranya adalah dengan membangun \emph{gateway} \cite{DaSilvaCampos2011} yang terhubung dengan \emph{sink node} dari WSN.


\section{Rumusan Masalah}
Bagaimana cara membangun sistem pengendalian WSN yang dapat dikendalikan secara jarak jauh dengan mudah dan dapat menginteroperabilitaskan WSN dan protokol Internet. Selain itu, sistem pengendalian yang dibangun juga harus dapat mengintegrasikan beberapa vendor WSN ke dalam sebuah \emph{gateway} yang sama dengan biaya yang murah.


\section{Batasan Masalah}
Batasan masalah pada penelitian ini adalah:
\begin{enumerate}
\item Penelitian ini difokuskan pada interoperabilitas beberapa \emph{vendor} WSN dan protokol Internet.
\item Tipe WSN yang digunakan dalam penelitian ini dibatasi dua buah.
\item Pengujian yang dilakukan hanya sebatas eksperimen dalam lingkup laboratorium.
\item Purwarupa yang dihasilkan akan diimplementasikan pada sebuah \emph{Access Point} (AP).
\end{enumerate}


\section{Tujuan Penelitian}
Tujuan penelitian ini adalah mempelajari kemungkinan pengembangan perangkat lunak yang akan ditanamkan ke dalam sebuah AP untuk difungsikan sebagai gateway sehingga mampu digunakan untuk mengintegrasikan jaringan WiFi dan beberapa protokol WSN ke jaringan internet dengan biaya yang murah.


\section{Manfaat Penelitian}
Dengan terhubungnya WSN ke jaringan internet dimungkinkan pengembangan aplikasi WSN yang dapat diakses melalui jaringan internet. Terhubungnya WSN ke jaringan internet akan membuka kemungkinan pengembangan layanan-layanan yang lebih beragam terutama layanan yang berbasis IP. Hal ini sejalan dengan perkembangan teknologi komunikasi yang menuju konvergensi penggunaan IP.

Selain itu, pengintegrasian gateway untuk WiFi dan WSN dalam satu piranti juga membuka peluang besar untuk memecahkan persoalan interoperabilitas perangkat keras dari berbagai \emph{vendor} dan kemudahan sistem.


\section{Keaslian Penelitian}
Penelitian ini tidak untuk menguji hipotesis baru melainkan merupakan pengembangan perangkat lunak yang akan ditanamkan ke dalam gateway sehingga mampu menghubungkan jaringan WiFi dan WSN ke jaringan internet. Penelitian ini akan meningkatkan fungsi AP menjadi gateway yang menghubungkan WiFi dan WSN dengan jaringan internet.


\section{Sistematika Penulisan}
\noindent
\textbf{BAB I : PENDAHULUAN}

Pada bab ini dijelaskan latar belakang, rumusan masalah, batasan, tujuan, manfaat, keaslian penelitian, dan sistematika penulisan.\\

\noindent
\textbf{BAB II : TINJAUAN PUSTAKA DAN LANDASAN TEORI}

Pada bab ini dijelaskan teori-teori dan penelitian terdahulu yang digunakan sebagai acuan dan dasar dalam penelitian.\\

\noindent
\textbf{BAB III : METODOLOGI PENELITIAN}

Pada bab ini dijelaskan metode yang digunakan dalam penelitian meliputi langkah kerja, pertanyaan penilitian, alat dan bahan, serta tahapan dan alur penelitian.\\

\noindent
\textbf{BAB IV : HASIL DAN PEMBAHASAN}

Pada bab ini dijelaskan hasil penelitian dan pembahasannya.\\

\noindent
\textbf{BAB V : KESIMPULAN DAN SARAN}

Pada bab ini ditulis kesimpulan akhir dari penelitian dan saran untuk pengembangan penelitian selanjutnya.\\

% Baris ini digunakan untuk membantu dalam melakukan sitasi
% Karena diapit dengan comment, maka baris ini akan diabaikan
% oleh compiler LaTeX.
\begin{comment}
\bibliography{daftar-pustaka}
\end{comment}
